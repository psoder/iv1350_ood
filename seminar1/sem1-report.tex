\documentclass[a4paper]{scrreprt}
\usepackage{fancyhdr}
\pagestyle{fancy}
\usepackage[english]{babel}
\usepackage[utf8]{inputenc}
\usepackage{graphicx}
\usepackage{url}
\usepackage{textcomp}
\usepackage{amsmath}
\usepackage{lastpage}
\usepackage{pgf}
\usepackage{wrapfig}
\usepackage{fancyvrb}

% Create header and footer
\headheight 27pt
\pagestyle{fancyplain}
\lhead{\footnotesize{Object-Oriented Design, IV1350}}
\chead{\footnotesize{Seminar 1 Solution}}
\rhead{}
\lfoot{}
\cfoot{\thepage\ (\pageref{LastPage})}
\rfoot{}

% Create title page
\title{Seminar 1}
\subtitle{Object-Oriented Design, IV1350}
\author{Pontus Söderlund, psoderlu@kth.se}
% \date{\today} Prints today's date
\date{March 30th}

\begin{document}

\maketitle

\tableofcontents %Generates the TOC

\chapter{Introduction}
    % A summary of the seminar task. Also, tell who you worked with when solving the task. 
    The goal of this seminar is to create a domain model (DM) for a retail store
    and a system sequence diagram (SSD) for a sale.

\chapter{Method}
    % Explain how you arrived at the solution presented in chapter \ref{sec:result}.
    % For example, one of the things to explain in the report for seminar one, is
    % how you used a category list and noun identification to find class candidates
    % for the domain model.
    
    \section{Domain Model}
        The process for creating a domain model can be borken down into five the 
        five steps listed bellow.

        \subsection{Identify class candidates}
            The first step is to read the specification and find all the nouns in
            in it. For the requirements in this lab some of the words found are
            listed below.

            \begin{itemize}
                \item Customer
                \item Goods
                \item Cashier
                \item Register
                \item Cash
                \item Change
                \item Program
                \item Reciept
            \end{itemize}

        \subsection{Use a category list}
            The second step is to use a category list to find even more class
            candidates.

            A category list is a tool to help find more class candidates by
            giving examples of categories that things not tought of in the first
            step belongs to. The project requiremients might for  
        
        \subsection{Filter class candidates}
            The next step is to filter out the class candidates that do not
            provide any additional value. It might be that one was too thorugh
            in previous steps or that the same information is in multiple places
            or something similar.
            
            In this case it 

        \subsection{Find attributes}
            After having filtered the class candidates I identified classes that
            were more suitable as attributes to classes. 
            
            Attributes are things that describe a class and not something that's
            part of it. A clear example of an attribute is a customer id since
            it is just an identifier for any given customer. An example of
            something that's not an attribute is a reciept since it can exist
            independently of a transaction.

        \subsection{Add associations}
            The last step is to att associations between the different classes.
            The program I used did not have any way (that I found) to create 

    \section{System Sequence Diagram}
            

\chapter{Result}
\label{sec:result}
    % This chapter shall contain the solution to the seminar tasks, together with
    % a short explanation. There are more detailed instructions in each
    % assignment.

    % Remember that code extracts and diagrams must be numbered, have caption and
    % be explained in the text, see figures \ref{fig:diag} and \ref{fig:code}

\begin{figure}[h!]
  \begin{center}
%    \includegraphics[scale=1.0]{sample-diag.png}
    \caption{A sample diagram to illustrate caption (this text), numbering and reference in text.}
    \label{fig:diag}
  \end{center}
\end{figure}

\begin{figure}[h!]
  \begin{center}
 %   \includegraphics[scale=0.7]{sample-code.png}
    \caption{A sample code extract.}
    \label{fig:code}
  \end{center}
\end{figure}

\chapter{Discussion}

Write an evaluation of the results presented in the result chapter (chapter \ref{sec:result} in this template). To get inspiration for the evaluation, read the assessment criteria in the documents \texttt{assessment-criteria-seminarX.pdf}, which are available on the \texttt{Assignments} page of the course web.

\end{document}
