\documentclass[a4paper]{scrreprt}
\usepackage{fancyhdr}
\pagestyle{fancy}

\usepackage[english]{babel}
\usepackage[utf8]{inputenc}
\usepackage{graphicx}
\usepackage{url}
\usepackage{textcomp}
\usepackage{amsmath}
\usepackage{lastpage}
\usepackage{pgf}
\usepackage{wrapfig}
\usepackage{fancyvrb}

\graphicspath{{./}}

% Create header and footer
\headheight 27pt
\pagestyle{fancyplain}
\lhead{\footnotesize{Object-Oriented Design, IV1350}}
\chead{\footnotesize{Seminar 1 Solution}}
\rhead{}
\lfoot{}
\cfoot{\thepage\ (\pageref{LastPage})}
\rfoot{}

% Create title page
\title{Seminar 1}
\subtitle{Object-Oriented Design, IV1350}
\author{Pontus Söderlund, psoderlu@kth.se}
% \date{\today} Prints today's date
\date{March 30th}

\begin{document}

\maketitle

\tableofcontents %Generates the TOC

\chapter{Introduction}
    % A summary of the seminar task. Also, tell who you worked with when solving the task. 
    The goal of this seminar is to create a domain model (DM) for a retail store
    and a system sequence diagram (SSD) for a sale.

\chapter{Method}
    % Explain how you arrived at the solution presented in chapter \ref{sec:result}.
    % For example, one of the things to explain in the report for seminar one, is
    % how you used a category list and noun identification to find class candidates
    % for the domain model.
    
    \section{Domain Model}
        The process for creating a domain model can be broken down into five the 
        five steps listed bellow.

        \subsection{Identify class candidates}
            The first step is to read the specification and find all the nouns in
            in it. For the requirements in this lab some of the words found are
            listed below.

            \begin{itemize}
                \item Customer
                \item Goods
                \item Cashier
                \item Register
                \item Cash
                \item Change
                \item Program
                \item Receipt
            \end{itemize}

        \subsection{Use a category list}
            The second step is to use a category list to find even more class
            candidates. A category list is a tool to help find more class candidates by
            giving examples of categories that things not thought of in the first
            step belongs to. I used it by going through all entries in the course book
            and asking myself the question, ''What could this mean in the context of a
            store?''. When I found something somewhat relevant I added it to the list of
            class candidates.
        
        \subsection{Filter class candidates}
            The next step is to filter out the class candidates that do not
            provide any additional value. It might be that one was too meticulous
            in previous steps or that the same information is in multiple places
            or something similar. In this case it meant that classes such as 
            ''Scanner'' and ''Storeshef'' were removed.

        \subsection{Find attributes}
            After having filtered the class candidates I identified classes that
            were more suitable as attributes to classes. Attributes are things 
            that describe a class and not something that's part of it. A clear example
            of an attribute is a customer id since it is just an identifier for any
            given customer. An example of something that's not an attribute is a
            receipt since it can exist independently of a transaction.

        \subsection{Add associations}
            The last step is to add associations between the different classes.
            The program I used did not have any way (that I found) to create
            the > (Class 1 Verb > Class 2) notation used in the lectures, I
            decided to use the directional association arrows instead. 

    \section{System Sequence Diagram}
        In order to create a ssd I went through all the steps in the process and 
        identified the main actor, the cashier, and the System (the register). After
        having done that I started going through all the steps and adding methods 
        for the steps that required it. When I came to a point where the system 
        communicated with another system I added it and created a method for data
        to be passed between them.

\chapter{Result}
    \label{sec:result}
    % This chapter shall contain the solution to the seminar tasks, together with
    % a short explanation. There are more detailed instructions in each
    % assignment.

    % Remember that code extracts and diagrams must be numbered, have caption and
    % be explained in the text, see figures \ref{fig:diag} and \ref{fig:code}
    
    \section{Domain Model}
        Figure \ref{fig:dm} shows a domain model for the requirements. The most interesting
        part of the model is ''Transaction'' because it is the reason for everything else
        existing. As mentioned above it could be argued that transaction should be split up
        into a transaction component containing, date and time and a payment class that
        contains details about the payment, but since I don't see any additional value being
        created from doing it I've decided not to. Again, the associations should not be
        directional, but since I didn't find a better solution to describe how the associations
        should be read they're there now.
        
        \begin{figure}[ht!]
            \begin{center}
                \includegraphics[scale=0.3]{dm1.png}
                \caption{The final domain model}
                \label{fig:dm}
            \end{center}
        \end{figure}
    
    \newpage
    
    \section{System Sequence Diagram}
        Figure \ref{fig:ssd} shows the system sequence diagram where ''Register'' is 
        the central object. The flow of the program is:
        
        \begin{enumerate}
            \item Scan items \\
                It is not relevant how the items are scanned, which is why there's no
                logic for if the customer has more items or whatever.
                
            \item Apply Discount \\
                Pretty straight forward, if the customer has a discount the customers
                id is sent away to a database of all discounts and the correct one(s) is
                applied. 
                
            \item Get the price \\
                The cashier gets the total price and tells the customer.
                
            \item Make the payment
                The customer makes the payment which is registered in the register and
                then the details of the transaction is sent to the accounting, inventory
                and printer.
                
        \end{enumerate}
        
        \begin{figure}[ht!]
            \begin{center}
                \includegraphics[scale=0.3]{ssd1.png}
                \caption{The system sequence diagram}
                \label{fig:ssd}
            \end{center}
        \end{figure}

\chapter{Discussion}
    % Write an evaluation of the results presented in the result chapter (chapter
    % \ref{sec:result} in this template). To get inspiration for the evaluation,
    % read the assessment criteria in the documents \texttt{assessment-criteria-seminarX.pdf},
    % which are available on the \texttt{Assignments} page of the course web.

    \section{Assesement Criterias}
        \subsection{Typical mistakes are to create a ''programmatic DM'' or a ''naïve DM''. Is the DM either of those?}
            I would not consider the DM naïve since it does not model flow but a
            snapshot of how everything fits together.

        \subsection{Are there ''spider-in-the-web'' classes?}
            I would consider ''Transaction'' to be a bit of a ''spider-in-the-web''.
            However, it is justified because ''Transaction'' is such an essential
            part. Nothing else serves any purpose without a transaction since the
            only point of a store is to facilitate transactions between parties.
            
        \subsection{Can you understand the DM? Is it a correct description of the problem domain?}
            I think that it is clear what the different classes represent and what
            associations they have with other classes. I might be biased though.

        \subsection{Does the DM have a reasonable number of classes? Are important classes missing?}
            I believe that there's a  number of classes for the given
            requirements, if anything there are too few.

        \subsection{Are there irrelevant classes?}
            Not that I can think of, some classes, like store would probably not
            be useful in a program, but since it's not a program it serves it
            purpose.

        \subsection{Do you agree with the choices between class and attribute?}
            Yes! It might have been better to have another class called 'Payment'
            so that the customer could make a payment and the cashier could
            receive payment, but I do feel that that that would be unnecessary 
            since it wouldn't add any additional information.

        \subsection{Is there a reasonable number of associations? Do you understand them? Are there classes missing associations? If so, is that a mistake?}
            No class is missing associations and I believe the associations are
            clear. More associations could be added, but I don't think that 
            would add any value.

        \subsection{Are naming conventions followed in DM and SSD?}
            The naming conventions follow traditional Java naming conventions.

        \subsection{Is UML used correctly in DM and SSD?}
            Besides the associations having directions I believe the diagramss
            adhere to the UML standard.

        \subsection{Are the correct objects included in the SSD?}
            I feel like they are, although I'm no expert. The only thing that
            might be missing is ''Customer'', but since the customer has nothing
            to do with the system it's not included.

\end{document}
